\documentclass[]{article} % for checking your page length
%\\documentclass[aip,rsi,preprint,graphicx]{revtex4-1} % for review purposes
\usepackage{amsmath, amssymb}
%\usepackage[numbers]{natbib}
\usepackage{graphicx}
\begin{document}
\section*{Revision notes}

Thank you for your comments on our paper. We have made changes to the article
according to your suggestions:\\[2ex]

\textit{1. In the abstract it is mentioned that the presented device is fit to supercede
“loudspeakers”. As we know, the loudspkears are not the only sources of
generating the droplets and multiple methods are known for this purpose; such as
pressure pulse, microfluid droplet generators, superhydrophobic syringe needles,
piezoelectric materials and so on. If the proposed device in this study can
supercede any other droplet generator in addition to the loudspeakers, then
mentioning “loudspeakers” repeatedly in the abstract does not make sense and the
abstract should be rewritten. If the proposed device can only supercede the
loudspeaker droplet generators, then one or two paragraphs should be added to
the introduction to notice this fact and discussions should be provided about
the reasons.}\\[2ex]

We agree. Unfortunately the page limit keeps us from elaborating on alternative
approaches in much detail. We have specified that we mean \textbf{``expedient''}
vibration sources, and we have qualified the next sentence in the abstract by adding
\textbf{``loudspeakers, which are typically used for prototyping such devices''}
to clarify that our device supercedes loudspeakers only.\\[2ex]

We have also added to the introduction a reference to some drop-on-demand
techniques you mention, specifically \textbf{``drop-on-demand approaches (e.g.
via pressure pulses, microfluidic devices etc.)''}. \\[2ex]

\textit{2. The figures in the paper must be presented consecutively in the order
        in which they are mentioned in the text. Therefore, Figure1 should not
        be presented in the first page since it is mentioned in the
        “CONTRUCTION” section for the first time. Arrows and textboxes could be
used to identify different elements of the device shown in Figure 1.}\\[2ex]

We have fixed the numbering and added arrows and text (and/or numbers) to the
figure elements.\\[2ex]

\textit{3. In general, the figures should be discussed more in the text. Mentioning just one or two sentences about them is not sufficient. The writer should walk the reader 
through different parts of his/her figure and highlight the reasons why he/she
is presenting that figure.}\\[2ex]

(See below)\\[2ex]

\textit{4. In my opinion, “CONSTRUCTION” is the most important section in this paper. However, following the steps mentioned in this section is not easy and straight forward. One figure should be added to this section to show the device preparation visually.
Suggestion: You may use 3 or 4 pictures for this purpose. First, you may show a
hard drive picture and use arrows and textboxes to identify what parts should be
removed in step “dismantle and cut”. In another picture, show the parts you are
going to remove in step “Expose coil leads”. Then, add another to identify where
you are going to drill and put the nozzle. And finally, “Figure 1” can be
presented in this figure as the last picture.}\\[2ex]

This is a good suggestion. We have cracked open another hard drive to label the
parts that need to be taken out, and thus added another figure which we then
refer to in the text. We condensed some text to make room for the extra figure;
any additional photos would push us over the page limit unfortunately.\\[2ex]

We have also fixed a few minor grammatical and clarity issues in the text.\\[2ex]

\textit{5. In section “OPERATION”, the most important question about the final results still remains and that is; how well the proposed device can handle generating different droplet sizes. In other words, more experimental results should be presented in form of a table in which flow rate, frequency and amplitude of the arm oscillations, and droplet size is listed. Then, a curve should be plotted in a figure to verify Equation2.
The current version of the paper only includes two different droplet sizes which
are shown in Figure4. This figure does not even provide the information about
the flow rates. Therefore, it is not possible to verify Equation2, based on the
only two available data in this paper.}\\[2ex]

This is a good comment. We have indicated the range of droplet sizes we have
generated in our lab (\textbf{``$\mathbf{0.1-1}\,$mm''}) and changed the figure to show
several frequency/flow rate conditions we used to make droplets, along with
their predicted sizes.

\end{document}
